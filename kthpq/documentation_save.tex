% Copyright (c) 2023 Isaac Ren. Licensed under the MIT license (see LICENSE.txt).

\documentclass[17pt, t, lualatex]{beamer}

%\begin{filecontents}{example.bib}
%@article{Riemann1859,
%  title={Über die {A}nzahl der {P}rimzahlen unter einer gegebenen {G}röße},
%  author={Riemann, G. F. B.},
%  journal={Monatsberichte der Königlichen Preußischen Akademie der Wissenschaften},
%  year={Nov. 1859},
%  pages={671--680},
%  address={Berlin}
%}
%\end{filecontents}

\title{A guide to the beamer package \texttt{kthpq}}
\date{November 7, 2023}
\institute[KTH]{KTH Royal Institute of Technology}
\author[Isaac Ren]{Isaac Ren}

\usepackage{amsmath,amssymb,mathtools}

% Probably load as late as possible
\usetheme[engine=lualatex, mathshape=sf]{kthpq}

\def\kthpq{\texttt{kthpq}}

\begin{document}

\inserttitlepage[center]

\begin{frame}
\tableofcontents
\end{frame}

\section{General information}

\insertsectionpage

\begin{frame}{Introduction}
\begin{itemize}
\item \emph{\kthpq{}} is a beamer template that implements the KTH Brand guidelines published in October 2023.
\begin{itemize}
\item The starting point is the provided PowerPoint template, although some liberties have been taken regarding beamer-specific elements, such as block environments and math formatting.
\end{itemize}
\item The name pays homage to \emph{PQ} (Promenadorquestern och med Baletten Paletten), KTH's student jazz band :)
\end{itemize}
\end{frame}

\begin{frame}[fragile=singleslide]{General description}
\kthpq{} is not too special, but there are some things to note:
\begin{itemize}
\item Slides either contain line patterns from KTH's Brands, or they can be added with the command \verb|\insertlines|. Some level of customization of the line patterns is possible; see next section.
\item The header and footer are rather minimal.
\begin{itemize}
\item The header shows the current section. This can be removed by adding \verb|\def\insertsection{}| in the preamble.
\item The footer can be customized with the commands \verb|\deffootline| and \verb|\setfootline|. There are three arguments, which correspond to the left, middle, and right parts of the footer.
\end{itemize}
\item Title and section slides contain neither the header nor the footer and are created by new commands \verb|\inserttitlepage| and \verb|\insertsectionpage|.
\end{itemize}
\end{frame}

\begin{frame}{In the background}
\begin{itemize}
\item All extra files are found in the \texttt{kthpq-files} directory.
\item \kthpq{} requires compiling with \emph{LuaLaTeX} in order to use the required font, \emph{Figtree}.
\begin{itemize}
\item For some reason the package \texttt{fontspec} used to load fonts is very verbose in its warnings, so it is called with the \texttt{quiet} option.
\end{itemize}
\item Figtree is actually quite limited in scope, so \kthpq{} also uses \emph{Fira Math} for sans serif math symbols and \emph{Bera Mono} for monospace fonts. The fonts are located in the \texttt{fonts/} directory.
\item KTH's visual elements are stored in the \texttt{figs/} directory.
\end{itemize}
\end{frame}

\begin{frame}[fragile=singleslide]{How to use}
\begin{itemize}
\item To use \kthpq{}, add the command \verb|\usepackage{kthpq}| in the preamble, probably towards the end.
\item There are two options you can add when loading the theme:
\begin{itemize}
\item \texttt{engine=lualatex} or \texttt{pdflatex}. The default and recommended engine for compiling with \kthpq{} is \texttt{lualatex}, which is the only way to get the recommended fonts Figtree and Georgia. The option \texttt{pdflatex} should be faster, but uses Helvetica and Bitstream Charter.
\item \texttt{mathshape=sf}, \texttt{rm}, or \texttt{custom}. This determines the shape used for math. The default is \texttt{sf}, sans-serif. \texttt{rm} corresponds to serif and \texttt{custom} means that no new math font is loaded (in case you want to load your own font).
\end{itemize}
\end{itemize}
\end{frame}

\begin{frame}{License}
\begin{itemize}
\item The source code of \kthpq{} is distributed under the MIT license. See \texttt{LICENSE.txt} for more information, although if you are just using this template for a presentation, you shouldn't need to worry about it :)
\end{itemize}
\end{frame}

\begin{frame}{What is not included?}
The following features are not included:
\begin{itemize}
\item Other types of headers or footers that are commonly used, such as \texttt{miniframes}, \texttt{split}, or \texttt{infolines}.
\item Frame subtitles and sidebars.
\item Article mode.
\item Different sized slides: the template is designed for $254 \times 143 \textnormal{mm}$ ($16\textnormal{:}9$) slides. The options \texttt{[17pt, lualatex]} are highly recommended, and the option \texttt{[t]} is recommended, following the guidelines.
\end{itemize}
\end{frame}

\begin{frame}{Reporting issues}
\begin{itemize}
\item This package is actively maintained by me (Isaac Ren), at least for the next few years (this is written in 2023).
\item If you encounter any issues or have any feature requests, please contact me by email. I'm not sure who will see this, so I won't write my email out here: it's \texttt{firstlast@kth.se}.
\end{itemize}
\end{frame}

\section{Customization and special commands}

\insertsectionpage

\begin{frame}{New commands}
\begin{itemize}
\item The title slide is inserted using the command \emph{\texttt{\textbackslash inserttitlepage[\textnormal{\textit{template}}]}}.
\begin{itemize}
\item The default template option for title slides is \texttt{center}. There is currently one other option, \texttt{left}, which is shown on the next slide.
\end{itemize}
\item The similar commands \emph{\texttt{\textbackslash insertsectionpage}} and \emph{\texttt{\textbackslash insertendpage}} insert section pages and end pages, respectively.
\item You can also insert lines in any \texttt{frame} environment by adding the command \emph{\texttt{\textbackslash insertlines}}. See further for more detailed explanations.
\item The commands \emph{\texttt{\textbackslash deffootline}} and \emph{\texttt{\textbackslash setfootline}} both take three arguments, which correspond to the left, middle, and right parts of the footer.
\end{itemize}
\end{frame}

\title{A left-aligned title}

\inserttitlepage[left]

\begin{frame}{Lines}
\insertlines
\begin{minipage}{.65\paperwidth}
\begin{itemize}
\item KTH's Brand guidelines allow users to insert certain abstract line patters on slides.
\item \kthpq{} comes with premade selections of lines. These are contained in macros of the form \emph{\texttt{\textbackslash lines\textnormal{\textit{cardinal}}}}, where \textit{cardinal} is one of the following: \texttt{northwest, northeast, east, southeast} (with some variants). These can be found in \texttt{kthpq-files/beamerinnerthemekthpq.sty}.
\item The command \texttt{\textbackslash insertlines} will insert \texttt{\textbackslash linesnortheast}. This is what you see on this slide.
\end{itemize}
\end{minipage}
\end{frame}

\begin{frame}[fragile=singleslide]{Smaller changes}
The following small features are also included and can be customized:
\begin{itemize}
\item The \verb|\emph| command has been changed to \emph{bold} instead of \textit{\usebeamercolor[fg]{emph} italic}. A new beamer color \verb|emph| has also been added, which is used by the command
\item All of KTH's recommended colors have been introduced and can be used to customize the color theme. See \texttt{kthpq-files/beamercolorthemekthpq.sty} for details.
\end{itemize}
\end{frame}

\section{Long section title to showcase different line wrapping options, and some math showcases}

\insertsectionpage


\begin{frame}{Prime numbers}
\begin{block}{Theorem}
There is no largest prime number.
\end{block}

\begin{proof}
\begin{enumerate}
\item Suppose $p$ were the largest prime number.
\item Consider $q := p! + 1$.
\item $q$ is not divisible by any number $\leq p$, so either it is prime or it is divisible by a prime number strictly greater than $p$.
\item Contradiction, so there is no largest prime number.
\end{enumerate}
\end{proof}
\end{frame}

\begin{frame}
\begin{exampleblock}{Examples}
The following are \emph{prime numbers}:
\[
2, 3, 5, 7, 11, 13, 17, 19, 23.
\]
\end{exampleblock}

\begin{alertblock}{Warning}
The number $1$ is not prime!
\end{alertblock}
\end{frame}

\begin{frame}
\begin{block}{A more complicated formula {\cite{Riemann1859}}}
Defining the \emph{logarithmic integral} and \emph{Riemann's prime-power counting function} respectively as
\[
\textnormal{li}(x) := \int_0^x \frac{\textnormal{d}t}{\ln t}
\qquad \textnormal{\usebeamercolor[fg]{normal text} and} \qquad
\Pi_0(x) := \frac{1}{2} \left( \sum_{p, n: p^n < x} \frac{1}{n} + \sum_{p, n: p^n \leq x} \frac{1}{n} \right) \textnormal{\usebeamercolor[fg]{normal text},}
\]
where $p$ spans prime numbers, we have the equation
\[
\Pi_0(x) = \textnormal{li}(x) - \sum_{\rho} \textnormal{li}(x^\rho) - \ln 2 + \int_x^\infty \frac{\textnormal{d}t}{t (t^2 - 1) \ln t} \textnormal{\usebeamercolor[fg]{normal text},}
\]
where $\rho$ spans the nontrivial zeros of the Riemann zeta function and $x > 1$.
\end{block}
\end{frame}


\begin{frame}
\insertlines
\frametitle{References}
\bibliographystyle{amsalpha}
\bibliography{example.bib}
\end{frame}

\insertendpage

\end{document}